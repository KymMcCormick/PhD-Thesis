\documentclass[]{article}
\usepackage{lmodern}
\usepackage{amssymb,amsmath}
\usepackage{ifxetex,ifluatex}
\usepackage{fixltx2e} % provides \textsubscript
\ifnum 0\ifxetex 1\fi\ifluatex 1\fi=0 % if pdftex
  \usepackage[T1]{fontenc}
  \usepackage[utf8]{inputenc}
\else % if luatex or xelatex
  \ifxetex
    \usepackage{mathspec}
  \else
    \usepackage{fontspec}
  \fi
  \defaultfontfeatures{Ligatures=TeX,Scale=MatchLowercase}
\fi
% use upquote if available, for straight quotes in verbatim environments
\IfFileExists{upquote.sty}{\usepackage{upquote}}{}
% use microtype if available
\IfFileExists{microtype.sty}{%
\usepackage{microtype}
\UseMicrotypeSet[protrusion]{basicmath} % disable protrusion for tt fonts
}{}
\usepackage[margin=1in]{geometry}
\usepackage{hyperref}
\hypersetup{unicode=true,
            pdftitle={Fitting the m-AFC eyewitness data to UV-SDT model},
            pdfauthor={Kym McCormick},
            pdfborder={0 0 0},
            breaklinks=true}
\urlstyle{same}  % don't use monospace font for urls
\usepackage{color}
\usepackage{fancyvrb}
\newcommand{\VerbBar}{|}
\newcommand{\VERB}{\Verb[commandchars=\\\{\}]}
\DefineVerbatimEnvironment{Highlighting}{Verbatim}{commandchars=\\\{\}}
% Add ',fontsize=\small' for more characters per line
\usepackage{framed}
\definecolor{shadecolor}{RGB}{248,248,248}
\newenvironment{Shaded}{\begin{snugshade}}{\end{snugshade}}
\newcommand{\KeywordTok}[1]{\textcolor[rgb]{0.13,0.29,0.53}{\textbf{#1}}}
\newcommand{\DataTypeTok}[1]{\textcolor[rgb]{0.13,0.29,0.53}{#1}}
\newcommand{\DecValTok}[1]{\textcolor[rgb]{0.00,0.00,0.81}{#1}}
\newcommand{\BaseNTok}[1]{\textcolor[rgb]{0.00,0.00,0.81}{#1}}
\newcommand{\FloatTok}[1]{\textcolor[rgb]{0.00,0.00,0.81}{#1}}
\newcommand{\ConstantTok}[1]{\textcolor[rgb]{0.00,0.00,0.00}{#1}}
\newcommand{\CharTok}[1]{\textcolor[rgb]{0.31,0.60,0.02}{#1}}
\newcommand{\SpecialCharTok}[1]{\textcolor[rgb]{0.00,0.00,0.00}{#1}}
\newcommand{\StringTok}[1]{\textcolor[rgb]{0.31,0.60,0.02}{#1}}
\newcommand{\VerbatimStringTok}[1]{\textcolor[rgb]{0.31,0.60,0.02}{#1}}
\newcommand{\SpecialStringTok}[1]{\textcolor[rgb]{0.31,0.60,0.02}{#1}}
\newcommand{\ImportTok}[1]{#1}
\newcommand{\CommentTok}[1]{\textcolor[rgb]{0.56,0.35,0.01}{\textit{#1}}}
\newcommand{\DocumentationTok}[1]{\textcolor[rgb]{0.56,0.35,0.01}{\textbf{\textit{#1}}}}
\newcommand{\AnnotationTok}[1]{\textcolor[rgb]{0.56,0.35,0.01}{\textbf{\textit{#1}}}}
\newcommand{\CommentVarTok}[1]{\textcolor[rgb]{0.56,0.35,0.01}{\textbf{\textit{#1}}}}
\newcommand{\OtherTok}[1]{\textcolor[rgb]{0.56,0.35,0.01}{#1}}
\newcommand{\FunctionTok}[1]{\textcolor[rgb]{0.00,0.00,0.00}{#1}}
\newcommand{\VariableTok}[1]{\textcolor[rgb]{0.00,0.00,0.00}{#1}}
\newcommand{\ControlFlowTok}[1]{\textcolor[rgb]{0.13,0.29,0.53}{\textbf{#1}}}
\newcommand{\OperatorTok}[1]{\textcolor[rgb]{0.81,0.36,0.00}{\textbf{#1}}}
\newcommand{\BuiltInTok}[1]{#1}
\newcommand{\ExtensionTok}[1]{#1}
\newcommand{\PreprocessorTok}[1]{\textcolor[rgb]{0.56,0.35,0.01}{\textit{#1}}}
\newcommand{\AttributeTok}[1]{\textcolor[rgb]{0.77,0.63,0.00}{#1}}
\newcommand{\RegionMarkerTok}[1]{#1}
\newcommand{\InformationTok}[1]{\textcolor[rgb]{0.56,0.35,0.01}{\textbf{\textit{#1}}}}
\newcommand{\WarningTok}[1]{\textcolor[rgb]{0.56,0.35,0.01}{\textbf{\textit{#1}}}}
\newcommand{\AlertTok}[1]{\textcolor[rgb]{0.94,0.16,0.16}{#1}}
\newcommand{\ErrorTok}[1]{\textcolor[rgb]{0.64,0.00,0.00}{\textbf{#1}}}
\newcommand{\NormalTok}[1]{#1}
\usepackage{graphicx,grffile}
\makeatletter
\def\maxwidth{\ifdim\Gin@nat@width>\linewidth\linewidth\else\Gin@nat@width\fi}
\def\maxheight{\ifdim\Gin@nat@height>\textheight\textheight\else\Gin@nat@height\fi}
\makeatother
% Scale images if necessary, so that they will not overflow the page
% margins by default, and it is still possible to overwrite the defaults
% using explicit options in \includegraphics[width, height, ...]{}
\setkeys{Gin}{width=\maxwidth,height=\maxheight,keepaspectratio}
\IfFileExists{parskip.sty}{%
\usepackage{parskip}
}{% else
\setlength{\parindent}{0pt}
\setlength{\parskip}{6pt plus 2pt minus 1pt}
}
\setlength{\emergencystretch}{3em}  % prevent overfull lines
\providecommand{\tightlist}{%
  \setlength{\itemsep}{0pt}\setlength{\parskip}{0pt}}
\setcounter{secnumdepth}{0}
% Redefines (sub)paragraphs to behave more like sections
\ifx\paragraph\undefined\else
\let\oldparagraph\paragraph
\renewcommand{\paragraph}[1]{\oldparagraph{#1}\mbox{}}
\fi
\ifx\subparagraph\undefined\else
\let\oldsubparagraph\subparagraph
\renewcommand{\subparagraph}[1]{\oldsubparagraph{#1}\mbox{}}
\fi

%%% Use protect on footnotes to avoid problems with footnotes in titles
\let\rmarkdownfootnote\footnote%
\def\footnote{\protect\rmarkdownfootnote}

%%% Change title format to be more compact
\usepackage{titling}

% Create subtitle command for use in maketitle
\providecommand{\subtitle}[1]{
  \posttitle{
    \begin{center}\large#1\end{center}
    }
}

\setlength{\droptitle}{-2em}

  \title{Fitting the m-AFC eyewitness data to UV-SDT model}
    \pretitle{\vspace{\droptitle}\centering\huge}
  \posttitle{\par}
    \author{Kym McCormick}
    \preauthor{\centering\large\emph}
  \postauthor{\par}
      \predate{\centering\large\emph}
  \postdate{\par}
    \date{16 April 2019}


\begin{document}
\maketitle

\begin{Shaded}
\begin{Highlighting}[]
\KeywordTok{library}\NormalTok{(}\StringTok{"plyr"}\NormalTok{, }\DataTypeTok{lib.loc=}\StringTok{"~/R/win-library/3.5"}\NormalTok{)}
\KeywordTok{library}\NormalTok{(}\StringTok{"dplyr"}\NormalTok{, }\DataTypeTok{lib.loc=}\StringTok{"~/R/win-library/3.5"}\NormalTok{)}
\end{Highlighting}
\end{Shaded}

\begin{verbatim}
## Warning: package 'dplyr' was built under R version 3.5.3
\end{verbatim}

\begin{verbatim}
## 
## Attaching package: 'dplyr'
\end{verbatim}

\begin{verbatim}
## The following objects are masked from 'package:plyr':
## 
##     arrange, count, desc, failwith, id, mutate, rename, summarise,
##     summarize
\end{verbatim}

\begin{verbatim}
## The following objects are masked from 'package:stats':
## 
##     filter, lag
\end{verbatim}

\begin{verbatim}
## The following objects are masked from 'package:base':
## 
##     intersect, setdiff, setequal, union
\end{verbatim}

\begin{Shaded}
\begin{Highlighting}[]
\KeywordTok{library}\NormalTok{(}\StringTok{"nnet"}\NormalTok{, }\DataTypeTok{lib.loc=}\StringTok{"~/R/win-library/3.5"}\NormalTok{)}
\end{Highlighting}
\end{Shaded}

\begin{verbatim}
## Warning: package 'nnet' was built under R version 3.5.3
\end{verbatim}

\begin{Shaded}
\begin{Highlighting}[]
\KeywordTok{library}\NormalTok{(}\StringTok{"MPTinR"}\NormalTok{, }\DataTypeTok{lib.loc=}\StringTok{"~/R/win-library/3.5"}\NormalTok{)}
\end{Highlighting}
\end{Shaded}

\begin{verbatim}
## Warning: package 'MPTinR' was built under R version 3.5.3
\end{verbatim}

\begin{Shaded}
\begin{Highlighting}[]
\KeywordTok{library}\NormalTok{(}\StringTok{"psych"}\NormalTok{, }\DataTypeTok{lib.loc=}\StringTok{"~/R/win-library/3.5"}\NormalTok{)}
\end{Highlighting}
\end{Shaded}

\begin{verbatim}
## Warning: package 'psych' was built under R version 3.5.3
\end{verbatim}

\begin{Shaded}
\begin{Highlighting}[]
\CommentTok{#library("tseries", lib.loc="~/R/win-library/3.5")}
\KeywordTok{library}\NormalTok{(}\StringTok{"DescTools"}\NormalTok{, }\DataTypeTok{lib.loc=}\StringTok{"~/R/win-library/3.5"}\NormalTok{)}
\end{Highlighting}
\end{Shaded}

\begin{verbatim}
## Warning: package 'DescTools' was built under R version 3.5.3
\end{verbatim}

\begin{verbatim}
## 
## Attaching package: 'DescTools'
\end{verbatim}

\begin{verbatim}
## The following objects are masked from 'package:psych':
## 
##     AUC, ICC, SD
\end{verbatim}

Forced choice accuracy is given by

\(P_C^{\langle k \rangle} = \int_{-\infty}^{\infty} g(x)F(x)^{n-1}dx\).

Step one is to find the best fitting UV-SDT model parameters to the
empirical data. I adaped the following code from Kellen Supplemental
script (2015)

\begin{Shaded}
\begin{Highlighting}[]
\CommentTok{#Empirical data for lineup size k = \{2,3,...,7,8\}}

\NormalTok{BMIDataCID <-}\StringTok{ }\KeywordTok{c}\NormalTok{(}\FloatTok{0.8862}\NormalTok{, }\FloatTok{0.7616}\NormalTok{, }\FloatTok{0.6737}\NormalTok{, }\FloatTok{0.6491}\NormalTok{, }\FloatTok{0.5606}\NormalTok{, }\FloatTok{0.5498}\NormalTok{, }\FloatTok{0.50276}\NormalTok{)}

\NormalTok{data_kafc <-}\StringTok{ }\KeywordTok{c}\NormalTok{(}\KeywordTok{rbind}\NormalTok{(BMIDataCID,}\DecValTok{1}\OperatorTok{-}\NormalTok{BMIDataCID))}\OperatorTok{*}\DecValTok{300}

\NormalTok{SDT_kafc <-}\StringTok{ }\ControlFlowTok{function}\NormalTok{(Q, data, param.names, n.params, tmp.env, lower.bound, upper.bound)\{}
   
\NormalTok{    e<-}\KeywordTok{vector}\NormalTok{(}
      \DataTypeTok{mode =} \StringTok{"numeric"}\NormalTok{,}
      \DataTypeTok{length =} \DecValTok{4}
\NormalTok{      )}

\NormalTok{    mu <-}\StringTok{ }\NormalTok{Q[}\DecValTok{1}\NormalTok{]}
\NormalTok{    ss <-}\StringTok{ }\NormalTok{Q[}\DecValTok{2}\NormalTok{]}
       
\NormalTok{    rank <-}\StringTok{ }\ControlFlowTok{function}\NormalTok{(i,k,}\DataTypeTok{mu=}\DecValTok{1}\NormalTok{,}\DataTypeTok{ss=}\DecValTok{1}\NormalTok{)\{}
\NormalTok{            f1 <-}\StringTok{ }\ControlFlowTok{function}\NormalTok{(x,i,k,mu,ss) \{}
              \KeywordTok{choose}\NormalTok{(k }\OperatorTok{-}\DecValTok{1}\NormalTok{, i}\OperatorTok{-}\DecValTok{1}\NormalTok{)}\OperatorTok{*}\KeywordTok{dnorm}\NormalTok{(x,mu,ss)}\OperatorTok{*}\KeywordTok{pnorm}\NormalTok{(x)}\OperatorTok{**}\NormalTok{(k}\OperatorTok{-}\NormalTok{i)}\OperatorTok{*}\NormalTok{(}\DecValTok{1}\OperatorTok{-}\KeywordTok{pnorm}\NormalTok{(x))}\OperatorTok{**}\NormalTok{(i}\OperatorTok{-}\DecValTok{1}\NormalTok{)    }
\NormalTok{              \}        }
\NormalTok{            tmp <-}\StringTok{ }\KeywordTok{vector}\NormalTok{(}
              \DataTypeTok{mode =} \StringTok{"numeric"}\NormalTok{,}
              \DataTypeTok{length =} \KeywordTok{length}\NormalTok{(i)}
\NormalTok{              )}
            \ControlFlowTok{for}\NormalTok{(ii }\ControlFlowTok{in} \DecValTok{1}\OperatorTok{:}\KeywordTok{length}\NormalTok{(i)) }
\NormalTok{              tmp[ii] <-}\StringTok{ }\KeywordTok{integrate}\NormalTok{(}
                \DataTypeTok{f =}\NormalTok{ f1,}
                \DataTypeTok{lower =} \OperatorTok{-}\OtherTok{Inf}\NormalTok{, }
                \DataTypeTok{upper =} \OtherTok{Inf}\NormalTok{, }
                \DataTypeTok{i =}\NormalTok{ i[ii],}
                \DataTypeTok{k =}\NormalTok{ k,}
                \DataTypeTok{mu =}\NormalTok{ mu,}
                \DataTypeTok{ss =}\NormalTok{ ss}
\NormalTok{                ) }\OperatorTok{$}\NormalTok{value}
            \KeywordTok{return}\NormalTok{(tmp)}
\NormalTok{    \}}
 

\NormalTok{    e[}\DecValTok{1}\OperatorTok{:}\DecValTok{2}\NormalTok{]  <-}\StringTok{ }\KeywordTok{rank}\NormalTok{(}\DecValTok{1}\OperatorTok{:}\DecValTok{2}\NormalTok{,}\DecValTok{2}\NormalTok{,}\DataTypeTok{mu=}\NormalTok{mu,}\DataTypeTok{ss=}\NormalTok{ss)    }
\NormalTok{    e[}\DecValTok{3}\OperatorTok{:}\DecValTok{4}\NormalTok{]  <-}\StringTok{ }\KeywordTok{c}\NormalTok{(}\KeywordTok{rank}\NormalTok{(}\DecValTok{1}\NormalTok{,}\DecValTok{3}\NormalTok{,}\DataTypeTok{mu=}\NormalTok{mu,}\DataTypeTok{ss=}\NormalTok{ss), }\DecValTok{1}\OperatorTok{-}\StringTok{ }\KeywordTok{rank}\NormalTok{(}\DecValTok{1}\NormalTok{,}\DecValTok{3}\NormalTok{,}\DataTypeTok{mu=}\NormalTok{mu,}\DataTypeTok{ss=}\NormalTok{ss))}
\NormalTok{    e[}\DecValTok{5}\OperatorTok{:}\DecValTok{6}\NormalTok{]  <-}\StringTok{ }\KeywordTok{c}\NormalTok{(}\KeywordTok{rank}\NormalTok{(}\DecValTok{1}\NormalTok{,}\DecValTok{4}\NormalTok{,}\DataTypeTok{mu=}\NormalTok{mu,}\DataTypeTok{ss=}\NormalTok{ss), }\DecValTok{1}\OperatorTok{-}\StringTok{ }\KeywordTok{rank}\NormalTok{(}\DecValTok{1}\NormalTok{,}\DecValTok{4}\NormalTok{,}\DataTypeTok{mu=}\NormalTok{mu,}\DataTypeTok{ss=}\NormalTok{ss) )}
\NormalTok{    e[}\DecValTok{7}\OperatorTok{:}\DecValTok{8}\NormalTok{]  <-}\StringTok{ }\KeywordTok{c}\NormalTok{(}\KeywordTok{rank}\NormalTok{(}\DecValTok{1}\NormalTok{,}\DecValTok{5}\NormalTok{,}\DataTypeTok{mu=}\NormalTok{mu,}\DataTypeTok{ss=}\NormalTok{ss), }\DecValTok{1}\OperatorTok{-}\StringTok{ }\KeywordTok{rank}\NormalTok{(}\DecValTok{1}\NormalTok{,}\DecValTok{5}\NormalTok{,}\DataTypeTok{mu=}\NormalTok{mu,}\DataTypeTok{ss=}\NormalTok{ss))}
\NormalTok{    e[}\DecValTok{9}\OperatorTok{:}\DecValTok{10}\NormalTok{] <-}\StringTok{ }\KeywordTok{c}\NormalTok{(}\KeywordTok{rank}\NormalTok{(}\DecValTok{1}\NormalTok{,}\DecValTok{6}\NormalTok{,}\DataTypeTok{mu=}\NormalTok{mu,}\DataTypeTok{ss=}\NormalTok{ss), }\DecValTok{1}\OperatorTok{-}\StringTok{ }\KeywordTok{rank}\NormalTok{(}\DecValTok{1}\NormalTok{,}\DecValTok{6}\NormalTok{,}\DataTypeTok{mu=}\NormalTok{mu,}\DataTypeTok{ss=}\NormalTok{ss) )}
\NormalTok{    e[}\DecValTok{11}\OperatorTok{:}\DecValTok{12}\NormalTok{] <-}\StringTok{ }\KeywordTok{c}\NormalTok{(}\KeywordTok{rank}\NormalTok{(}\DecValTok{1}\NormalTok{,}\DecValTok{7}\NormalTok{,}\DataTypeTok{mu=}\NormalTok{mu,}\DataTypeTok{ss=}\NormalTok{ss), }\DecValTok{1}\OperatorTok{-}\StringTok{ }\KeywordTok{rank}\NormalTok{(}\DecValTok{1}\NormalTok{,}\DecValTok{7}\NormalTok{,}\DataTypeTok{mu=}\NormalTok{mu,}\DataTypeTok{ss=}\NormalTok{ss) )}
\NormalTok{    e[}\DecValTok{13}\OperatorTok{:}\DecValTok{14}\NormalTok{] <-}\StringTok{ }\KeywordTok{c}\NormalTok{(}\KeywordTok{rank}\NormalTok{(}\DecValTok{1}\NormalTok{,}\DecValTok{8}\NormalTok{,}\DataTypeTok{mu=}\NormalTok{mu,}\DataTypeTok{ss=}\NormalTok{ss), }\DecValTok{1}\OperatorTok{-}\StringTok{ }\KeywordTok{rank}\NormalTok{(}\DecValTok{1}\NormalTok{,}\DecValTok{8}\NormalTok{,}\DataTypeTok{mu=}\NormalTok{mu,}\DataTypeTok{ss=}\NormalTok{ss) ) }
    \CommentTok{#add this last line if you have k=8 kAFC data}

    
\NormalTok{    LL <-}\StringTok{ }\OperatorTok{-}\KeywordTok{sum}\NormalTok{(data[data}\OperatorTok{!=}\DecValTok{0}\NormalTok{]}\OperatorTok{*}\KeywordTok{log}\NormalTok{(e[data}\OperatorTok{!=}\DecValTok{0}\NormalTok{]))}
    \KeywordTok{return}\NormalTok{(LL)}
\NormalTok{\}}


\NormalTok{fit_kafc <-}\StringTok{ }\KeywordTok{fit.mptinr}\NormalTok{(}
  \DataTypeTok{data =}\NormalTok{ data_kafc, }
  \DataTypeTok{objective =}\NormalTok{ SDT_kafc, }
  \DataTypeTok{param.names =} \KeywordTok{c}\NormalTok{(}\StringTok{"mu"}\NormalTok{, }\StringTok{"sigma"}\NormalTok{), }
  \DataTypeTok{categories.per.type =} \KeywordTok{c}\NormalTok{(}\DecValTok{2}\NormalTok{,}\DecValTok{2}\NormalTok{,}\DecValTok{2}\NormalTok{,}\DecValTok{2}\NormalTok{,}\DecValTok{2}\NormalTok{,}\DecValTok{2}\NormalTok{,}\DecValTok{2}\NormalTok{), }
  \DataTypeTok{lower.bound =} \KeywordTok{c}\NormalTok{(}\DecValTok{0}\NormalTok{,}\FloatTok{0.1}\NormalTok{), }
  \DataTypeTok{upper.bound =} \OtherTok{Inf}\NormalTok{, }
  \DataTypeTok{n.optim =} \DecValTok{5}\NormalTok{,}
  \DataTypeTok{show.messages =} \OtherTok{FALSE}
\NormalTok{  )}

\NormalTok{fit_kafc}
\end{Highlighting}
\end{Shaded}

\begin{verbatim}
## $goodness.of.fit
##   Log.Likelihood G.Squared df   p.value
## 1      -1276.376  2.608118  5 0.7601314
## 
## $information.criteria
##        AIC     BIC
## 1 6.608118 17.9075
## 
## $model.info
##   rank.fisher n.parameters n.independent.categories
## 1           2            2                        7
## 
## $parameters
##       estimates lower.conf upper.conf
## mu    1.3318149  1.2426560  1.4209739
## sigma 0.6336813  0.3190277  0.9483348
## 
## $data
## $data$observed
##        [,1]  [,2]   [,3]  [,4]   [,5]  [,6]   [,7]   [,8]   [,9]  [,10]
## [1,] 265.86 34.14 228.48 71.52 202.11 97.89 194.73 105.27 168.18 131.82
##       [,11]  [,12]   [,13]   [,14]
## [1,] 164.94 135.06 150.828 149.172
## 
## $data$predicted
## list()
## 
## 
## $fitting.runs
##           Min.   1st Qu.    Median      Mean   3rd Qu.      Max.
## [1,] -1276.376 -1276.376 -1276.376 -1276.376 -1276.376 -1276.376
\end{verbatim}

Below is the code to simulate CIDs from the best fitting UV-SDT model
parameters

\begin{Shaded}
\begin{Highlighting}[]
\NormalTok{mu=}\FloatTok{1.3093966}
\NormalTok{ss=}\FloatTok{0.5939441}
\NormalTok{simulated <-}\StringTok{ }\KeywordTok{vector}\NormalTok{(}\StringTok{"numeric"}\NormalTok{)}
\NormalTok{rank <-}\StringTok{ }\ControlFlowTok{function}\NormalTok{(i,k,}\DataTypeTok{mu=}\FloatTok{1.3093966}\NormalTok{,}\DataTypeTok{ss=}\FloatTok{0.5939441}\NormalTok{)\{}
\NormalTok{            f1 <-}\StringTok{ }\ControlFlowTok{function}\NormalTok{(x,i,k,mu,ss) \{}
              \KeywordTok{choose}\NormalTok{(k }\OperatorTok{-}\DecValTok{1}\NormalTok{, i}\OperatorTok{-}\DecValTok{1}\NormalTok{)}\OperatorTok{*}\KeywordTok{dnorm}\NormalTok{(x,mu,ss)}\OperatorTok{*}\KeywordTok{pnorm}\NormalTok{(x)}\OperatorTok{**}\NormalTok{(k}\OperatorTok{-}\NormalTok{i)}\OperatorTok{*}\NormalTok{(}\DecValTok{1}\OperatorTok{-}\KeywordTok{pnorm}\NormalTok{(x))}\OperatorTok{**}\NormalTok{(i}\OperatorTok{-}\DecValTok{1}\NormalTok{)    }
\NormalTok{              \}        }
\NormalTok{            tmp <-}\StringTok{ }\KeywordTok{vector}\NormalTok{(}
              \DataTypeTok{mode =} \StringTok{"numeric"}\NormalTok{,}
              \DataTypeTok{length =} \KeywordTok{length}\NormalTok{(i)}
\NormalTok{              )}
            \ControlFlowTok{for}\NormalTok{(ii }\ControlFlowTok{in} \DecValTok{1}\OperatorTok{:}\KeywordTok{length}\NormalTok{(i)) }
\NormalTok{              tmp[ii] <-}\StringTok{ }\KeywordTok{integrate}\NormalTok{(}
                \DataTypeTok{f =}\NormalTok{ f1,}
                \DataTypeTok{lower =} \OperatorTok{-}\OtherTok{Inf}\NormalTok{, }
                \DataTypeTok{upper =} \OtherTok{Inf}\NormalTok{, }
                \DataTypeTok{i =}\NormalTok{ i[ii],}
                \DataTypeTok{k =}\NormalTok{ k,}
                \DataTypeTok{mu =}\NormalTok{ mu,}
                \DataTypeTok{ss =}\NormalTok{ ss}
\NormalTok{                ) }\OperatorTok{$}\NormalTok{value}
            \KeywordTok{return}\NormalTok{(tmp)}
\NormalTok{\}}

\NormalTok{simulated[}\DecValTok{1}\NormalTok{]  <-}\StringTok{ }\KeywordTok{rank}\NormalTok{(}\DecValTok{1}\NormalTok{,}\DecValTok{2}\NormalTok{,}\DataTypeTok{mu=}\NormalTok{mu,}\DataTypeTok{ss=}\NormalTok{ss)    }
\NormalTok{simulated[}\DecValTok{2}\NormalTok{]  <-}\StringTok{ }\KeywordTok{rank}\NormalTok{(}\DecValTok{1}\NormalTok{,}\DecValTok{3}\NormalTok{,}\DataTypeTok{mu=}\NormalTok{mu,}\DataTypeTok{ss=}\NormalTok{ss)}
\NormalTok{simulated[}\DecValTok{3}\NormalTok{]  <-}\StringTok{ }\KeywordTok{rank}\NormalTok{(}\DecValTok{1}\NormalTok{,}\DecValTok{4}\NormalTok{,}\DataTypeTok{mu=}\NormalTok{mu,}\DataTypeTok{ss=}\NormalTok{ss)}
\NormalTok{simulated[}\DecValTok{4}\NormalTok{]  <-}\StringTok{ }\KeywordTok{rank}\NormalTok{(}\DecValTok{1}\NormalTok{,}\DecValTok{5}\NormalTok{,}\DataTypeTok{mu=}\NormalTok{mu,}\DataTypeTok{ss=}\NormalTok{ss)}
\NormalTok{simulated[}\DecValTok{5}\NormalTok{] <-}\StringTok{ }\KeywordTok{rank}\NormalTok{(}\DecValTok{1}\NormalTok{,}\DecValTok{6}\NormalTok{,}\DataTypeTok{mu=}\NormalTok{mu,}\DataTypeTok{ss=}\NormalTok{ss)}
\NormalTok{simulated[}\DecValTok{6}\NormalTok{] <-}\StringTok{ }\KeywordTok{rank}\NormalTok{(}\DecValTok{1}\NormalTok{,}\DecValTok{7}\NormalTok{,}\DataTypeTok{mu=}\NormalTok{mu,}\DataTypeTok{ss=}\NormalTok{ss)}
\NormalTok{simulated[}\DecValTok{7}\NormalTok{] <-}\StringTok{ }\KeywordTok{rank}\NormalTok{(}\DecValTok{1}\NormalTok{,}\DecValTok{7}\NormalTok{,}\DataTypeTok{mu=}\NormalTok{mu,}\DataTypeTok{ss=}\NormalTok{ss)}
\NormalTok{simulated}
\end{Highlighting}
\end{Shaded}

\begin{verbatim}
## [1] 0.8698738 0.7703444 0.6913704 0.6269855 0.5733839 0.5280067 0.5280067
\end{verbatim}

I realised then that I need to add in a G squared test of some sort (not
sure if it is correct). I don't really understand how this differs from
the loglikelihood test used in the fitting\ldots{}.

\begin{Shaded}
\begin{Highlighting}[]
\NormalTok{sim_kafc <-}\StringTok{ }\KeywordTok{c}\NormalTok{(}\KeywordTok{rbind}\NormalTok{(simulated,}\DecValTok{1}\OperatorTok{-}\NormalTok{simulated))}\OperatorTok{*}\DecValTok{300}
\NormalTok{matriz <-}\StringTok{ }\KeywordTok{cbind}\NormalTok{(sim_kafc,data_kafc)}
\NormalTok{matriz}
\end{Highlighting}
\end{Shaded}

\begin{verbatim}
##        sim_kafc data_kafc
##  [1,] 260.96215   265.860
##  [2,]  39.03785    34.140
##  [3,] 231.10333   228.480
##  [4,]  68.89667    71.520
##  [5,] 207.41113   202.110
##  [6,]  92.58887    97.890
##  [7,] 188.09566   194.730
##  [8,] 111.90434   105.270
##  [9,] 172.01518   168.180
## [10,] 127.98482   131.820
## [11,] 158.40202   164.940
## [12,] 141.59798   135.060
## [13,] 158.40202   150.828
## [14,] 141.59798   149.172
\end{verbatim}

\begin{Shaded}
\begin{Highlighting}[]
\KeywordTok{GTest}\NormalTok{(matriz)}
\end{Highlighting}
\end{Shaded}

\begin{verbatim}
## 
##  Log likelihood ratio (G-test) test of independence without
##  correction
## 
## data:  matriz
## G = 1.7409, X-squared df = 13, p-value = 0.9999
\end{verbatim}

Next, I quickly wrote up some code to test the Block-Marschak
Inequalities for the best fitting simulated CIDs and the empirical CIDs

\begin{Shaded}
\begin{Highlighting}[]
\NormalTok{BMI <-}\StringTok{ }\ControlFlowTok{function}\NormalTok{(e)\{}
  
\NormalTok{BMI1 <-}\StringTok{ }\KeywordTok{vector}\NormalTok{(}\StringTok{"numeric"}\NormalTok{)}
\NormalTok{BMI2 <-}\StringTok{ }\KeywordTok{vector}\NormalTok{(}\StringTok{"numeric"}\NormalTok{)}
\NormalTok{BMI3 <-}\StringTok{ }\KeywordTok{vector}\NormalTok{(}\StringTok{"numeric"}\NormalTok{)}
\NormalTok{BMI4 <-}\StringTok{ }\KeywordTok{vector}\NormalTok{(}\StringTok{"numeric"}\NormalTok{)}
\NormalTok{BMI5 <-}\StringTok{ }\KeywordTok{vector}\NormalTok{(}\StringTok{"numeric"}\NormalTok{)}
\NormalTok{BMI6 <-}\StringTok{ }\KeywordTok{vector}\NormalTok{(}\StringTok{"numeric"}\NormalTok{)}

\ControlFlowTok{for}\NormalTok{ (b }\ControlFlowTok{in} \DecValTok{1}\OperatorTok{:}\DecValTok{6}\NormalTok{) \{}
\NormalTok{  BMI1[b] <-}\StringTok{ }\NormalTok{e[b]}\OperatorTok{-}\NormalTok{e[b}\OperatorTok{+}\DecValTok{1}\NormalTok{]}
\NormalTok{\}}
\ControlFlowTok{for}\NormalTok{ (b }\ControlFlowTok{in} \DecValTok{2}\OperatorTok{:}\DecValTok{6}\NormalTok{) \{}
\NormalTok{  BMI2[b}\OperatorTok{-}\DecValTok{1}\NormalTok{] <-}\StringTok{ }\NormalTok{e[b}\OperatorTok{-}\DecValTok{1}\NormalTok{]}\OperatorTok{-}\DecValTok{2}\OperatorTok{*}\NormalTok{e[b]}\OperatorTok{+}\NormalTok{e[b}\OperatorTok{+}\DecValTok{1}\NormalTok{]}
\NormalTok{\}}
\ControlFlowTok{for}\NormalTok{ (b }\ControlFlowTok{in} \DecValTok{3}\OperatorTok{:}\DecValTok{6}\NormalTok{) \{}
\NormalTok{  BMI3[b}\OperatorTok{-}\DecValTok{2}\NormalTok{] <-}\StringTok{ }\NormalTok{e[b}\OperatorTok{-}\DecValTok{2}\NormalTok{]}\OperatorTok{-}\DecValTok{3}\OperatorTok{*}\NormalTok{e[b}\OperatorTok{-}\DecValTok{1}\NormalTok{]}\OperatorTok{+}\DecValTok{3}\OperatorTok{*}\NormalTok{e[b]}\OperatorTok{-}\NormalTok{e[b}\OperatorTok{+}\DecValTok{1}\NormalTok{]}
\NormalTok{\}}
\ControlFlowTok{for}\NormalTok{ (b }\ControlFlowTok{in} \DecValTok{4}\OperatorTok{:}\DecValTok{6}\NormalTok{) \{}
\NormalTok{  BMI4[b}\OperatorTok{-}\DecValTok{3}\NormalTok{] <-}\StringTok{ }\NormalTok{e[b}\OperatorTok{-}\DecValTok{3}\NormalTok{]}\OperatorTok{-}\DecValTok{4}\OperatorTok{*}\NormalTok{e[b}\OperatorTok{-}\DecValTok{2}\NormalTok{]}\OperatorTok{+}\DecValTok{6}\OperatorTok{*}\NormalTok{e[b}\OperatorTok{-}\DecValTok{1}\NormalTok{]}\OperatorTok{-}\DecValTok{4}\OperatorTok{*}\NormalTok{e[b]}\OperatorTok{+}\NormalTok{e[b}\OperatorTok{+}\DecValTok{1}\NormalTok{]}
\NormalTok{\}}
\ControlFlowTok{for}\NormalTok{ (b }\ControlFlowTok{in} \DecValTok{5}\OperatorTok{:}\DecValTok{6}\NormalTok{) \{}
\NormalTok{  BMI5[b}\OperatorTok{-}\DecValTok{4}\NormalTok{] <-}\StringTok{ }\NormalTok{e[b}\OperatorTok{-}\DecValTok{4}\NormalTok{]}\OperatorTok{-}\DecValTok{5}\OperatorTok{*}\NormalTok{e[b}\OperatorTok{-}\DecValTok{3}\NormalTok{]}\OperatorTok{+}\DecValTok{10}\OperatorTok{*}\NormalTok{e[b}\OperatorTok{-}\DecValTok{2}\NormalTok{]}\OperatorTok{-}\DecValTok{10}\OperatorTok{*}\NormalTok{e[b}\OperatorTok{-}\DecValTok{1}\NormalTok{]}\OperatorTok{+}\DecValTok{5}\OperatorTok{*}\NormalTok{e[b]}\OperatorTok{-}\NormalTok{e[b}\OperatorTok{+}\DecValTok{1}\NormalTok{]}
\NormalTok{\}}
\ControlFlowTok{for}\NormalTok{ (b }\ControlFlowTok{in} \DecValTok{6}\NormalTok{) \{}
\NormalTok{  BMI6[b}\OperatorTok{-}\DecValTok{5}\NormalTok{] <-}\StringTok{ }\NormalTok{e[b}\OperatorTok{-}\DecValTok{5}\NormalTok{]}\OperatorTok{-}\DecValTok{6}\OperatorTok{*}\NormalTok{e[b}\OperatorTok{-}\DecValTok{4}\NormalTok{]}\OperatorTok{+}\DecValTok{15}\OperatorTok{*}\NormalTok{e[b}\OperatorTok{-}\DecValTok{3}\NormalTok{]}\OperatorTok{-}\DecValTok{20}\OperatorTok{*}\NormalTok{e[b}\OperatorTok{-}\DecValTok{2}\NormalTok{]}\OperatorTok{+}\DecValTok{15}\OperatorTok{*}\NormalTok{e[b}\OperatorTok{-}\DecValTok{1}\NormalTok{]}\OperatorTok{-}\DecValTok{6}\OperatorTok{*}\NormalTok{e[b]}\OperatorTok{+}\NormalTok{e[b}\OperatorTok{+}\DecValTok{1}\NormalTok{]}
\NormalTok{\}}
\NormalTok{BMI <-}\StringTok{ }\KeywordTok{c}\NormalTok{(BMI1,BMI2,BMI3,BMI4,BMI5,BMI6)}
\KeywordTok{return}\NormalTok{(BMI)}
\NormalTok{\}}


\NormalTok{BMI.simulated <-}\StringTok{ }\KeywordTok{BMI}\NormalTok{(}\KeywordTok{c}\NormalTok{(}\DecValTok{1}\NormalTok{,simulated))}

\NormalTok{empirical <-}\StringTok{ }\KeywordTok{c}\NormalTok{(}\DecValTok{1}\NormalTok{,}\FloatTok{0.8866}\NormalTok{,    }\FloatTok{0.7564}\NormalTok{, }\FloatTok{0.6726}\NormalTok{, }\FloatTok{0.6494}\NormalTok{, }\FloatTok{0.5503}\NormalTok{, }\FloatTok{0.5446}\NormalTok{)}
\CommentTok{# I've added in the CID for k = 1 so that we can calculate the full BMI}
\NormalTok{BMI.empirical <-}\StringTok{ }\KeywordTok{BMI}\NormalTok{(empirical)}
\NormalTok{BMI.comparison <-}\StringTok{ }\NormalTok{(}\KeywordTok{cbind}\NormalTok{(BMI.simulated,BMI.empirical))}
\NormalTok{BMI.comparison}
\end{Highlighting}
\end{Shaded}

\begin{verbatim}
##       BMI.simulated BMI.empirical
##  [1,]   0.130126163        0.1134
##  [2,]   0.099529404        0.1302
##  [3,]   0.078973998        0.0838
##  [4,]   0.064384907        0.0232
##  [5,]   0.053601579        0.0991
##  [6,]   0.045377216        0.0057
##  [7,]   0.030596759       -0.0168
##  [8,]   0.020555407        0.0464
##  [9,]   0.014589091        0.0606
## [10,]   0.010783328       -0.0759
## [11,]   0.008224362        0.0934
## [12,]   0.010041352       -0.0632
## [13,]   0.005966316       -0.0142
## [14,]   0.003805762        0.1365
## [15,]   0.002558966       -0.1693
## [16,]   0.004075036       -0.0490
## [17,]   0.002160554       -0.1507
## [18,]   0.001246797        0.3058
## [19,]   0.001914483        0.1017
## [20,]   0.000913757       -0.4565
## [21,]   0.001000726        0.5582
\end{verbatim}


\end{document}
